
% [handout]{beamer} アニメーションを考慮しない配布用の資料作成
\documentclass[dvipdfmx, 12pt]{beamer}
\usepackage{pgfpages}
\usepackage{bxdpx-beamer}

%%%% PDFのしおりの文字化け対策 %%%%
\usepackage{atbegshi}
\ifnum 42146=\euc"A4A2
\AtBeginShipoutFirst{\special{pdf:tounicode EUC-UCS2}}
\else
\AtBeginShipoutFirst{\special{pdf:tounicode 90ms-RKSJ-UCS2}}
\fi
%%%%

%%%% スライドの見た目 %%%%
% Theme, Color, Font は一覧からお好みのものをどうぞ
% http://deic.uab.es/~iblanes/beamer_gallery/
\usetheme{Singapore} % Theme
\usecolortheme{default} % Color
\usefonttheme{professionalfonts} % Font
\setbeamertemplate{headline}{ % headerの設定
  \leavevmode\vskip10pt % 上のマージン(100番教室用の余白設定)
}
\setbeamertemplate{navigation symbols}{} % ナビゲーションの無効化
\setbeamertemplate{footline}{% footerの設定
  \scriptsize\bfseries % フォント
  \leavevmode\raggedleft % 右寄せ
  \insertframenumber{} / \inserttotalframenumber % ページ番号 / ページ総数
  \hspace*{5ex} % 右のマージン(100番教室用の余白設定)
  \vskip15pt % 下のマージン(100番教室用の余白設定)
}

\AtBeginSection[]{ % Do nothing for \section*
  \frame{\tableofcontents[currentsection, hideallsubsections]} % 目次スライド
}
%%%%

%%%% フォントの設定 %%%%
\usepackage[utf8]{inputenc} % UTF-8
\usepackage[T1]{fontenc} % 8bit font
\usepackage{lmodern} % Latin Modern font
\usepackage{otf} % otf (font)
\usepackage{bm} % bold math
\renewcommand{\kanjifamilydefault}{\gtdefault} % fontfamily: Gothic
\renewcommand{\familydefault}{\sfdefault} % 英語フォント
\usepackage{txfonts}
\usepackage{xcolor} % \textcolor{color}{text}
%%%%

% %% ----- math -----
% \usepackage{amsmath, amssymb}
% \renewcommand{\~}{$\sim$}
% %% ----- figure -----
% \usepackage{xcolor}
% \usepackage{here}
% \usepackage[dvipdfmx, hiresbb]{graphicx}
% %% ----- table -----
% \usepackage{booktabs}
% \catcode63=\active \def?{\phantom{0}} % ? -> ' '
% \usepackage{tabularx} \newcolumntype{Y}{>{\centering\arraybackslash}X}
% \usepackage{multirow}
% %% ----- programing -----
% \usepackage{verbatim}
% \newcommand{\code}[1]{ \texttt{\detokenize{#1}} }
% %% --- border ---
% \usepackage{fancybox, ascmac}
% %% ----- tikz -----
% \usepackage{tikz} % xcolor -> graphicx -> tikz
% \usetikzlibrary{fit}
% calc, positioning, quotes, topaths, scopes, spy
% matrix, graphs, graph.standard, trees, chains, automata, mindmap, er, calendar
% shapes.(geometric|symbols|arrows|multipart|callouts|misc)?
% arrows, patterns, fadings, shadings, shadows, backgrounds
% circuits.(logic.US|ee.IEC), lindenmayersystems, folding, petri, svg.path
% decorations.(pathmorphing|pathreplacing|markings|footprints|shapes|text|fractals)?
% datavisualization(.formats.functions)?
% intersections, plothandlers, plotmarks, through

\title{LaTeX + Beamer でプレゼン作成}
\subtitle{}
% % 一人の場合
\author{発表 太郎}
\institute{◯◯学科 ◯◯研究室}
% % 複数人の場合
% \author{太郎\inst{1} \and 花子\inst{2}}
% \institute{\inst{1} 太郎の所属 \and \inst{2} 花子の所属}
\date{}

\begin{document}

% ==============================================================================
\begin{frame}[plain]
  \frametitle{}
  \titlepage %表紙
\end{frame}

% ==============================================================================
\frame{\tableofcontents[hideallsubsections]} % 目次スライド
\section{見出し}
\begin{frame}
  \frametitle{見出し}

  \begin{itemize}
    \item Hello, \LaTeX.
    \item 日本語の印字テスト.
    \item $e^{i\pi} = -1$
  \end{itemize}

\end{frame}

% ==============================================================================
\section{見出し2}
\begin{frame}
  \frametitle{見出し2}

\end{frame}

% % columns
% \begin{columns}
%   \begin{column}{.4\textwidth}
%     ...
%   \end{column}
%   \begin{column}{.6\textwidth}
%     ...
%   \end{column}
% \end{columns}
%
% % block, alertblock, exampleblock
% \begin{block}{見出し}
%   ...
% \end{block}
%
% % Overlay Specification
% \pause
% \begin{enumerate}
%   \item<1-> 1コマ目以降表示
%   \item<1-2> 1-2コマ目のみ表示
%   \item<2-3> 2-3コマ目のみ表示
% \end{enumerate}
% \uncover<1,3>{1コマ目と3コマ目のみ表示、スペースを占有する。}
% \only<2>{2コマ目のみ表示、スペースを占有しない。}
% \only<2|handout:0>{2コマ目のみ表示、handoutモードで非表示。}
%
% % Text animations (revealing one item per click)
% \begin{itemize}[<+->]
%   \item ...
% \end{itemize}

\end{document}
